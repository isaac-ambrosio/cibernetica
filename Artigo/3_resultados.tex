\section{Resultados}

Após realizadas as configurações necessárias para a execução do código, é iniciado o treinamento do modelo referência rtdetr\_r50vd\_6x\_coco.

Com aproximadamente 1h de treinamento, são executadas as 100 épocas pré-definidas do modelo. Para este projeto está sendo adotado o AP50 (\textit{Average Precision} com IoU = 0.50), ou seja, uma detecção é considerada correta (True Positive) se a sobreposição entre a caixa predita e a caixa \textit{ground truth} for pelo menos 50\%.

\begin{figure}[H]
    \centering
    \includegraphics[width=1\linewidth]{imagens/AP50Buriti.png}
    \caption{Evolução de AP50 ao longo de 100 épocas para a classe Buriti}
    \label{fig:ap50buriti}
\end{figure}


Após executar as 100 épocas, o maior valor de AP50 é o da época 48 (AP=0,994). Para realizar o gráfico de AP50 por número de épocas, foi realizado um \textit{regex} no arquivo de log em busca dos termos chave e salvando o maior valor de AP \cref{fig:ap50buriti}.  

Percebe-se uma tendência rápida de estabilização do valor de AP50 já na época de número 20, mantendo acima de 0,80 até o final do treinamento.

Foi realizado o mesmo treinamento com a classe de Palmito, também com 100 épocas. Ao final, o maior valor de AP50 para esta classe foi de 0,715 na época 89. Neste caso, percebe-se uma diferença quanto à estabilização, tendo uma maior variação entre épocas do treinamento \cref{fig:ap50palmito}.

Esse comportamento pode estar relacionado ao baixo número de imagens rotuladas para esta classe.


\begin{figure}
    \centering
    \includegraphics[width=1\linewidth]{imagens/AP50Palmito.png}
    \caption{Evolução de AP50 ao longo de 100 épocas para a classe Palmito}
    \label{fig:ap50palmito}
\end{figure}




 
Por fim, para testar a precisão do modelo referência treinado (rtdetr\_r50vd\_6x\_coco), foi feito um teste com duas imagens presente no dataset. Sendo a \cref{fig:resultado_buriti} a identificação de Buritis e a \cref{fig:resultado_palmito} o resultado da identificação de Palmito Juçara.

Como pode ser observado, o modelo conseguiu identificar com precisão os indivíduos arbóreos presentes na imagem, mesmo aqueles de menor tamanho.

No caso do Buriti, no canto superior há um indivíduo parcialmente, porém não foi identificado pelo detector. Já o Palmito Juçara, como o valor de AP50 apresentou um valor inferior, alguns indivíduos na imagem não foram rotulados pelo detector.

\begin{figure}[H]
    \centering
    \includegraphics[width=1\linewidth]{imagens/results_buriti.jpg}
    \caption{Resultado da identificação de Buriti}
    \label{fig:resultado_buriti}
\end{figure}

\begin{figure}
    \centering
    \includegraphics[width=1\linewidth]{imagens/results_palmito.jpg}
    \caption{Resultado da identificação de Palmito Juçara}
    \label{fig:resultado_palmito}
\end{figure}

Na segunda abordagem, realizando o treinamento das duas classes simultaneamente, foi obtido um valor máximo de AP50 de 0.893 ao final das 100 épocas \ref{fig:resultado_comb}.

\begin{figure}[H]
    \centering
    \includegraphics[width=1\linewidth]{imagens/Combinado.png}
    \caption{Evolução de AP50 ao longo de 100 épocas para as classes Buriti e Palmito combinadas}
    \label{fig:resultado_comb}
\end{figure}



O mesmo teste, nas mesmas imagens \ref{fig:resultado_buriti} e \ref{fig:resultado_palmito}, foi realizado e demonstrado a seguir (Figura \ref{fig:resultado_combinado}:


\begin{figure}[H]
    \centering
    \includegraphics[width=1\linewidth]{imagens/results_comb.jpg}
    \caption{Resultado da identificação de Buriti, quando combinado com Palmito}
    \label{fig:resultado_combinado}
\end{figure}

Como observado, diferentemente quando treinado isoladamente, agora o modelo não conseguiu identificar com êxito todas os quatros indivíduos presentes na imagem, o que demonstra uma pequena queda de confiança no modelo, de acordo com o menor valor de AP50. 

Com a execução do modelo RT-DETR no dataset, foi realizada uma tabela comparativa dos valores de AP50 para as duas abordagens (\cref{tab:resultados_ap50}).

\begin{table}[H]
\centering
\caption{Comparativo de AP50 entre classes}
\label{tab:resultados_ap50}
\begin{tabular}{c|c|c|c}
\toprule
\textbf{Classe} & \textbf{AP50\_Max} & \textbf{AP50\_e50}  & \textbf{AP50\_e100}\\
\midrule
Buriti & 0.994 & 0.976 & 0.911 \\
Palmito & 0.715 & 0.613 & 0.587 \\
\midrule
Buriti e Palmito & 0.893 & 0.545 & 0.527 \\
 \bottomrule

\end{tabular}
\end{table}