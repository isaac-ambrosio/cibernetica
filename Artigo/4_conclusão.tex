\section{Conclusão}

Neste Projeto buscou-se realizar a implementação e avaliação de métodos de\textit{ Deep Learning}, mais especificamente o método RT-DETR, para identificação de espécies arbóreos presentes no Cerrado.

Utilizando um \textit{dataset} obtido com auxílio de drone, foi possível comparar os resultados obtidos para as espécies de Buriti e Palmito Juçara. Foram encontrados bons resultados no treinamento, indicando uma capacidade de detecção desses indivíduos mesmo em uma vegetação densa.

Ao realizar o treinamento com mais de uma classe, observou-se uma queda na confiança AP50 do detector, o que já era esperado, demonstrando que ao adicionar mais classes, o modelo deve aprender a diferenciar entre várias categorias, aumentando a complexidade e o potencial de erros.

Os dados utilizados neste projetos estão presentes no repositório Github: https://github.com/isaac-ambrosio/cibernetica.
